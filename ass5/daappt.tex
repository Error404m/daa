
\documentclass[conference]{IEEEtran}
\IEEEoverridecommandlockouts
% The preceding line is only needed to identify funding in the first footnote. If that is unneeded, please comment it out.
\usepackage{cite}
\usepackage{amsmath,amssymb,amsfonts}
\usepackage{algorithmic}
\usepackage{graphicx}
\usepackage{textcomp}
\usepackage{xcolor}
\usepackage{tabularx}

\usepackage{pgfplots}
\pgfplotsset{width=8.5cm,compat=3}
\usepackage{listings}
\usepackage{cleveref}
\renewcommand{\lstlistingname}{Code}

\lstdefinestyle{chstyle}{%
backgroundcolor=\color{gray!12},
basicstyle=\ttfamily\small,
commentstyle=\color{green!60},
keywordstyle=\color{magenta},
stringstyle=\color{blue!50!red},
frame=lr
}

\def\BibTeX{{\rm B\kern-.05em{\sc i\kern-.025em b}\kern-.08em
    T\kern-.1667em\lower.7ex\hbox{E}\kern-.125emX}}
\begin{document}

\title{Maximum Amount of Gold Collected\\

}

\author{\IEEEauthorblockN{Ishneet Sethi}
\IEEEauthorblockA{\textit{(IIT2019237)} \\
}
\and

\IEEEauthorblockN{Chandramani Kumar}
\IEEEauthorblockA{\textit{(IIT2019238)}\\\
\textit{.}\\
\\
IV Semester B.Tech in Information Technology \\
Indian Institute of Information Technology Allahabad, Prayagraj(U.P.), India
}\hline
\and
\IEEEauthorblockN{Mrityunjaya Tiwari}
\IEEEauthorblockA{\textit{(IIT2019239)} \\
}
}
\maketitle







\begin{abstract}
In this paper, we have devised an algorithm to find the maximum amount of gold collected  by a miner while starting from any cell of the first column and traversing a matrix of N X M size whose each cell contains the amount of gold.


\end{abstract}



\section{Introduction}
Bottom up (tabulation) dynamic programming approach has been used to implement this algorithm.\\
Dynamic Programming (DP) is an algorithmic technique for solving an optimization problem by breaking it down into simpler subproblems and utilizing the fact that the optimal solution to the overall problem depends upon the optimal solution to its subproblems.\\
Bottom approach is so called because one starts with the base result and gets to the topmost desired result.\\\\\\

This report further contains -\\
II. Algorithm Design\\
III Algorithm and Illustration\\
III. Algorithm Analysis\\
IV. Conclusion\\\\

\section{ALGORITHM DESIGN}



\subsection{Algorithm-1(Recursion):}

Here the idea is to recursively get the maximum amount of gold collected by the miner while starting from each cell of the first column and traversing the matrix in all possible ways thereby updating the maximum amount of gold collected by recurring  in the three directions as mentioned in the question i.e right,right\textunderscore up and right\textunderscore down.\\
Finally, the value the variable max\textunderscore gold\textunderscore amt will contain gives us the required answer.


\subsection{Algorithm-2(Dynamic Programming) :}

The idea is to store the amount of gold that can be collected in each cell if the miner starts collecting from that cell.\\\\

To do this we are implementing the bottom up approach of dynamic programming. We are creating a 2D matrix of the same size as the given matrix to store what we discussed before. For each row of the last column of the created matrix, value will be equal to that of the given matrix as no move can be possible from those cells. For the rest of the columns, we will store the sum of the value of the present cell in the given matrix and maximum of values of the cells in the created matrix which are feasible to be reached on moving from that cell. Finally the required result will be found on traversing through each row and finding the maximum of the first column.



\section{ALGORITHM ANALYSIS}

\subsection{Algorithm 1 Pseudo Code}\label{AA}



\begin{lstlisting}[caption=Algorithm 1, style=chstyle, language=python]

//=====================================
            //--PSEUDO CODE--//
//====================================

function main()
	srand(time(0))
	get R and C
	    for  i ← 1 to R do
		 for  j ← 1 to C do
      matrix[i][j] ← rand
           for  i ← 1 to R do
                ans ← max(ans,
                maxm_Gold_amnt(R,C,i,1,matrix))

       print ans


function maxm_Gold_amnt(R, C, row, col, matrix)
         if (col == C+1 || 
         row == 0 || row == R+1) then
		return 0;

      long long int right ← 						                       maxm_Gold_amnt(row,col+1)
      long long int  					   right_up ←  maxm_Gold_amnt(row-1,col+1)
	long long int 						   right_down ← maxm_Gold_amnt(row+1,col+1)
		
     return matrix[row][col] +
                     max( right, 
                     max(right_up, right_down))



function max(a,b)
	if (a>b) then
		return a
	return a



\end{lstlisting}



\subsection{Algorithm 2 Pseudo Code}\label{AA}



\begin{lstlisting}[caption=Algorithm 2, style=chstyle, language=python]

//=====================================
            //--PSEUDO CODE--//
//====================================

function main( )
srand(time(0))
get R and C
for  i ← 1 to R do
	for  j ← 1 to C do
		matrix[i][j] ← rand

print maxm_Gold_amnt(R, C,matrix)





function maxm_Gold_amnt(R, C,matrix)
	long long int dp[R][C]  ← {0}

      for  i ← 1 to R do
		dp[i][C]  ← matrix[i][C]

	for i ← C-1 to 1 do
            for j ← 1 to R  do 
     if  (i  ← 1) then
          if ( i+1 <= R ) then
	dp[i][j] ← matrix[i][j] +   			            max(dp[i][j+1],dp[i+1][j+1])
		          else
 	 dp[i][j] ← matrix[i][j] + dp[i][j+1]

      else if ( i ← R) then
            dp[i][j]  ← matrix[i][j] + 			       max(dp[i][j+1],dp[i-1][j+1])
        
      else
            long long int  m ← 			        max(dp[i+1][j+1],dp[i-1][j+1])

            dp[i][j] ←  matrix[i][j] + 				          max(dp[i][j+1],m);
     
      long long int maxm ← -1

	for i ← 1 to R do
		if ( dp[i][1] > maxm) then
		      maxm ←  dp[i][1]


      return maxm


function max(a,b)
	if (a>b) then
		return a
	return a


\end{lstlisting}


\section{ILLUSTRATION}

\begin{itemize}
\item INPUT:
        R = 3, C = 3\\
        m[3][3] = \begin{Bmatrix}
4 & 5 & 6\\
7 & 10 & 20\\
10 & 20 & 12
\end{Bmatrix}
\item Avoid combining SI and CGS units, such as current in amperes and magnetic field in oersteds. This often leads to confusion because equations do not balance dimensionally. If you must use mixed units, clearly state the units for each quantity that you use in an equation.
\item First, we will create a 2d matrix dp[3][3] with each cell initialised with 0.\\\\
Initially dp[][] matrix be :\\

dp[3][3] = \begin{Bmatrix}
0 & 0 & 0\\
0 & 0 & 0\\
0 & 0 & 0
\end{Bmatrix}
 
 
 \item Now, using algorithm 2 for last column cells  the updated matrix will be :\\
dp[3][3] = \begin{Bmatrix}
0 & 0 & 6\\
0 & 0 & 20\\
0 & 0 & 12
\end{Bmatrix}


\item Now, for all other column cells algorithm 2 implementation is demonstrated as :\\\\

dp[0][1] = matrix[0][1] + max(6,20)\\
	= matrix[0][1] + 20\\
	= 5 + 20 = 25\\\\

dp[1][1] = matrix[1][1] + max(6,max(20,12))\\
	= matrix[1][1] + 20\\
	= 10 + 20 = 30\\\\

dp[2][1] = matrix[2][1] + max(20,12)\\
	= matrix[2][1] + 20\\
	= 20 + 20 = 40\\\\

Updated matrix dp[][] will be :\\
dp[3][3] = \begin{Bmatrix}
0 & 25 & 6\\
0 & 30 & 20\\
0 & 40 & 12\\\\
\end{Bmatrix}



\item dp[0][0] = matrix[0][0] + max(25,30)\\
	= matrix[0][0] + 30\\
	= 4 + 30 = 34\\\\

dp[1][0] = matrix[1][0] + max(25,max(30,40))\\
	= matrix[1][0] + 40\\
	= 7 + 40 = 47\\\\

dp[2][0] = matrix[2][0] + max(30,40)\\
	= matrix[2][0] + 40\\
	= 10 + 40= 50\\\\
	
	Finally the dp[][] matrix will be :\\
dp[3][3] =  \begin{Bmatrix}
34 & 25 & 6\\
47 & 30 & 20\\
50 & 40 & 12\\\\
\end{Bmatrix}


\item Finally, the required maximum amount of gold collected will be maximum of 34, 47 and 50, i.e 50.


\item Output: 50.



\end{itemize}
\section{ALGORITHM ANALYSIS}
\subsection{Time Complexity}
\section*{Recursion}

In this approach, we are following some steps :\\
1)  Traversing each row for updating maxm.\\
       ↳  In each iteration of the loop, we are recurring thrice through m-1 columns to get the maximum amount of gold collected if started with that cell of the first column.\\\\

Clearly, time complexity gets affected by the above two steps.\\
Let's suppose step 1 time complexity be T1 \\\\
So, the Time complexity of the whole program:
      \[ (T) = T1 \]

Assuming 1 unit time for comparison and assignment.\\

 Now, calculating T1 :\\
No of rows traversed = N\\
So, T1 = N*(T’)\\
  where , T’ =  Time complexity for recursive function\\
                    = width of the recursive tree\\
	      = \( 3\wedge (m-1) \) \\
Therefore the time complexity would be:

\[O(N*3\wedge (M-1))\]
     

\bigskip



\section*{Dynamic Programming}


In this approach, we are following some steps :\\
1) Traversing each row for the last column cells’ value.\\
2) Traversing each row and column except the last one to store the maximum gold value in each cell of the dp[][] matrix.\\
3) Traversing each row of the first column to find the maximum among them.\\\\

Clearly, time complexity gets affected by the above three steps.\\
Let's suppose step 1 time complexity be T1,  step 2 time complexity be T2 and that of step 3 be T3\\
So, the Time complexity of the whole program\\\\ (T)=T1+T2+T3\\

Now, calculating T1 :\\\\

No of rows traversed = N\\
So, T1 = N*(T’)\\
where , T’ = 1 unit for assigning operation

\[T1 = O(N)\]



 Now, calculating T2 :\\
No of rows traversed = N\\
No of columns traversed  = M-1\\
So, T2 = O((N)*(M-1))*(T’’)\\
where T’’ = 1 unit time for each comparisons made and for assignment operation
\[T2 = O(N*(M-1)))\]


Now, calculating T3 :\\
 No of rows traversed = N\\
 So, T3 = N*(T’’’)\\
        where , T’’’ = 1 unit time for each comparisons \\
                         made and for assignment operation
  
 
 \[T3 = O(N)\]

Hence, T = O(N) + O(N*(M-1)) + O(N)\\
`      T	 ~ O(N*M)\\\\

Best Case Complexity: O(N*M)\\\\

Worst Case Complexity: O(N*M)\\\\

Average Case Complexity:  O(N*M)\\\\

\subsection{Space Complexity}
\section*{Recursion algorithm}
Our memory complexity is determined by the number of return statements because each function call will be stored on the program stack. To generalize, a recursive function's memory complexity is O(recursion depth).Hence in this case space complexity would be O(M).\bigskip

\section*{Dynamic Programming}


Since we are creating a 2D matrix of the same size as of the original matrix, i.e N X M, so we are using N*M space .Hence, the space complexity of the algorithm 2 will be O(N*M).\\\\

For the analysis of the algorithm, we break the code snippet into several segments and count the number of addition, subtraction, multiplication, division, assign, comparison and by considering all those operations as 1 unit of operation, we will try to find the overall time complexity of the algorithm by multiplying the total  number of operations into the number of times the code segment is running in the algorithm. For the space complexity, we tend to count the total amount of space taken by that algorithm to implement and then find a relation between the amount of space required and the input. The time and space complexity analysis for various steps is as shown below .\\

\bigskip
\section{ Aposteriori analysis}\bigskip
\begin{tabularx}{0.4\textwidth} { | >{\raggedright\arraybackslash}X | >{\centering\arraybackslash}X | >{\raggedleft\arraybackslash}X | }
   \hline
   
Row size & Column size & Time(in milliseconds)
 \\
   \hline
   
10 & 100 & 10.270 \\
   \hline
10 & 500 & 11.804 \\
   \hline

10 & 1000  &   10.796 \\
   \hline


10 &
3000 &
        11.842 \\
   \hline

10 &
5000
 &       10.618 \\
   \hline


10 &
8000
    &     9.827 \\
   \hline


10 &
10000
   &      10.369 \\
   \hline


10 &
15000
    &      9.775 \\
   \hline

       100 &
            10
         & 9.816 \\
   \hline
   
       500 &
                   10
         & 10.551 \\
   \hline
   
   
       1000
         &  10
         & 11.834 \\
   \hline
   
       5000
         &  10
         & 10.593 \\
   \hline
   
8000
         &  10
         & 11.899   \\
   \hline
   
      10000
         &  10
         & 10.379 \\
   \hline
   
      15000
        &   10
        &  18.520

  \\
   \hline
\end{tabularx}

\section*{Figure1: Time complexity graph for aposteriori analysis. ( Keeping N constant and M on X-axis )
}

\begin{tikzpicture}
\begin{axis}[
    title={},
    xlabel={M},
    ylabel={Time(ms)},
     xmin=0, xmax=15000,
    ymin=0, ymax=20,
    xtick={5000,10000,15000},
    ytick={0,5,10,15,20},
    legend pos=north west,
    ymajorgrids=true,
    grid style=dashed,
]

\addplot[
    color=blue,
    mark=square,
    ]
    coordinates {
    (0,10)(500,12)(1000,11)(3000,11)(5000,10)(8000,5)(10000,10.4)(15000,9.8)
    };
   
    
\end{axis}
\end{tikzpicture}



\section*{Figure2: Time complexity graph for aposteriori analysis. ( Keeping M constant and N on X-axis )}

\begin{tikzpicture}
\begin{axis}[
    title={},
    xlabel={N},
    ylabel={Time(ms)},
     xmin=0, xmax=15000,
    ymin=0, ymax=20,
    xtick={5000,10000,15000},
    ytick={0,5,10,15,20},
    legend pos=north west,
    ymajorgrids=true,
    grid style=dashed,
]

\addplot[
    color=blue,
    mark=square,
    ]
    coordinates {
    (0,10)(500,10.8)(1000,12)(5000,11)(7000,12)(10000,10.5)(10000,10.4)(15000,18.520)
    };
   
    
\end{axis}
\end{tikzpicture}

\bigskip
In the above graphs, the value ‘N’ and ‘M’ i.e., row and columns (randomly generated in their graph) are considered along the x-axis (M in Fig:1 and N in Fig:2 ) and time along the y-axis. The graph represents Time v/s M (or N).


\section{Conclusion}
The algorithms discussed in this paper can be used to find the maximum amount of gold collected by a miner while starting from any cell of the first column. traversing a matrix of N X M size whose each cell contains the amount of gold.In the matter of comparison between these two algorithms both approaches had the same space complexity. But, in Approach 1, time complexity would be higher as compared to Algorithm 2. So we can conclude that Dynamic Programming Solution i.e the Algorithm 2 is optimal and efficient.

\section{REFERENCES}
[1] https://en.wikipedia.org/wiki/Dynamic_programming\\\\

[2] https://www.geeksforgeeks.org/tabulation-vs-memoization/\\\\
	
[3] https://www.ideserve.co.in/learn/time-and-space-complexity-of-recursive-algorithms\\\\

[4] https://www.geeksforgeeks.org/auxiliary-space-recursive-functions/\\\\


\end{document}
