\documentclass[conference]{IEEEtran}
\IEEEoverridecommandlockouts
% The preceding line is only needed to identify funding in the first footnote. If that is unneeded, please comment it out.
\usepackage{cite}
\usepackage{amsmath,amssymb,amsfonts}
\usepackage{algorithm}
\usepackage{algorithmic}
\usepackage{graphicx}
\usepackage{textcomp}
\usepackage{xcolor}
\usepackage[noend]{algpseudocode}
\usepackage{tabularx}
\usepackage{mathtools}
\usepackage{pgfplots}
\pgfplotsset{width=8.5cm,compat=3}
\usepackage{listings}
\usepackage{cleveref}
\renewcommand{\lstlistingname}{Code}


\lstdefinestyle{chstyle}{%
backgroundcolor=\color{gray!12},
basicstyle=\ttfamily\small,
commentstyle=\color{blue!60},
keywordstyle=\color{magenta},
stringstyle=\color{blue!50!red},
frame=lrb
}

\def\BibTeX{{\rm B\kern-.05em{\sc i\kern-.025em b}\kern-.08em
    T\kern-.1667em\lower.7ex\hbox{E}\kern-.125emX}}
\begin{document}

\title{Count of Different Palindromic Substring\\

}

\author{\IEEEauthorblockN{Ishneet Sethi}
\IEEEauthorblockA{\textit{(IIT2019237)} \\
}
\and

\IEEEauthorblockN{Chandramani Kumar}
\IEEEauthorblockA{\textit{(IIT2019238)}\\\
\textit{.}\\
\\
IV Semester B.Tech in Information Technology \\
Indian Institute of Information Technology Allahabad, Prayagraj(U.P.), India
}\hline
\and
\IEEEauthorblockN{Mrityunjaya Tiwari}
\IEEEauthorblockA{\textit{(IIT2019239)} \\
}
}
\maketitle







\begin{abstract}
In this paper, we have devised an algorithm to count the number of non-empty sub strings that are palindromes provided two substrings are different if they occur at different positions in the given string.


\end{abstract}



\section{Introduction}
Bottom up (tabulation) dynamic programming approach has been used to implement this algorithm.
Dynamic Programming (DP) is an algorithmic technique for solving an optimization problem by breaking it down into simpler subproblems and utilizing the fact that the optimal solution to the overall problem depends upon the optimal solution to its subproblems.\\
Bottom approach is so called because one starts with the base result and gets to the topmost desired result.\bigskip

A sub string is any continuous sequence of characters in the string. A string is said to be palindrome, if the reverse of the string is the same as itself.\bigskip


This report further contains -\\
II. Algorithm Design\\
III. Algorithm and Illustration\\
IV. Algorithm Analysis\\
V. Conclusion\\
VI. Reference



\bigskip\section{ALGORITHM DESIGN}



\bigskip\subsection{Algorithm-1(Recursion):}

If length of string is 2 then we check both character are same or not and return 0/1 accordingly.If length is 1 we return 1 since its a palindrome..In recursion if i crosses j then it will be an invalid condition conditions we need to return 0.Else If str[i..j] is palindrome increment count by 1 and check for rest palindromic substring (i, j-1), (i+1, j)and remove common palindrome substring (i+1, j-1) Else if substring is NOT PALINDROME We check for rest palindromic substrings (i, j-1)and (i+1, j) remove common palindrome substring (i+1 , j-1).

\bigskip\subsection{Algorithm-2(Two-loop Strategy) :}

The idea is to consider all the substrings and check if they form a palindrome or not.
To do this we are implementing the two for-loops to consider all the substrings. Basically the first for-loop determines the start index index of the substring and the second for-loop determines the end index of the substring.Hence one by one considering all the substrings and checking if it’s a palindrome we increment the count.
To check if the substring is a palindrome or not we use the ispalindrome function explained already in the above mentioned approaches.
Finally the total number of palindrome substrings present in the string is printed.


\bigskip\subsection{Algorithm-3(Dynamic Programming) :}

The idea is to identify the cells corresponding to a substring which are palindromic.\\

To do this we are implementing the bottom up approach of dynamic programming. We are creating a 2D matrix with string length as the number of rows and columns. The upper half of the matrix separated by the main diagonal corresponds to all the substrings possible for the given string. Further, the cells on each diagonal in the upper half correspond to substring of the same length. Basically, the cells on the main diagonal correspond to those substring whose lengths are 1 and similarly length increases as diagonal no increase. We will be marking the cells on the main diagonal of the matrix as true because the single letter string is palindrome by default. For the cells on the second diagonal in the upper half, we will check if both characters in the string are the same (marked cell true) or not (marked cell false). For rest of the diagonals, we will check similarity of first and last character of the substring and also check if the string
between the first and the last character is palindrome ( marked true) or not (marked false). That in between string palindrome check can be done by looking the cell at left-down. Meanwhile, we will be incrementing the counter if cells’ value comes out to be true.

\bigskip\subsection{Algorithm-4(Manacher’s Algorithm) :}

The idea is to count the number of even length and odd length palidromic substrings separately; the idea for each is basically the same, but there are enough variations that it is worth separating the two entirely.\\
The idea is that we don't have need to repeat calculations on elements we have "seen before". If (when going from left to right) we have seen elements at index R and we are at index i, we know there is some index k s.t. R is the right endpoint of k's longest palindrom centered around it. Thus, one can check that we can map i to some index j (where j is on the other half of k's longest palindrome and is the same distance from the left endpoint instead of the right endpoint), we can use the longest palindrome length at index j to give a lower bound on the longest palindrome at i (notice that the characters around both of them within the longest palindrome k is contained in are the same - hence we already have computed everything we need unless if we extend pass the boundary).

\bigskip\section{ALGORITHM AND ILUSTRATION}

\bigskip\subsection{Algorithm 1 Pseudo Code}\label{AA}



\begin{lstlisting}[caption=Algorithm 1, style=chstyle, language=c]

//=====================================
            //--PSEUDO CODE--//
//====================================

function main( )
get string 
long long int n <- string.size()
long long int res

res <- CountPS(string,0,n-1)
print res


function CountPS(string,l,r)
   if(l>r)
	  return 0;
	if(l==r)
     return 1;
   long long int cnt ← 0
      if (ispalindrome(str,l,r))
	    cnt <- 1

        cnt <- cnt + CountPS(string,l+1,r)


       cnt <-  cnt + CountPS(string,l,r-1)
      cnt <-  cnt - CountPS(string,l+1,r-1)
      
     return cnt





function ispalindrome(string s,int i,int j)
    i  <-  0, j  <-  n-1
    while(i<=j)
	  if(s[i++]!=s[j--])
	  return 0;
    end

   return 1;



\end{lstlisting}



\bigskip\subsection{Algorithm 2 Pseudo Code}\label{AA}



\begin{lstlisting}[caption=Algorithm 2, style=chstyle, language=c++]

//=====================================
            //--PSEUDO CODE--//
//====================================

function ispalindrome(string s,int i,int j)
    i <- 0, j <- n-1
    while(i<=j)
	  if(s[i++]!=s[j--])
	  return 0;
    end
   return 1

function main( )
get string 
long long int n <- string.size()
long long int count
count <- 0
for i <- 0 to n
  for j <- to n
     if(ispalindrome(s,i,j)
        count <- count+1 
  print count




\end{lstlisting}

\bigskip\subsection{Algorithm 3 Pseudo Code}\label{AA}



\begin{lstlisting}[caption=Algorithm 3, style=chstyle, language=c++]

//=====================================
            //--PSEUDO CODE--//
//====================================

function main( )
get string
long long int res
res  <-  count_palindromes(string)
print res




function count_palindromes(string)
	long long int cnt  <-  0

	long long int len  <-  string.size()

	vector<vector<bool> >  dp(len, vector<bool> (len))

	for gap ← 0  to len-1  do
            for i  <-  0 , j  <-  gap  to len-1  do 
     if  (gap  <-   0) then
          dp[i][j]  <-    true
     else if ( gap <-   1) then
            if ( string[i] == string[j] )
	           dp[i][j]  <-  true
            else
	           dp[i][j]  <-  false     
     else
           if ( (string[i] == string[j]) and
           (dp[i + 1][j - 1]) )
            dp[i][j]  <-    true
           else
	       dp[i][j]  <-  false
     
     
   if ( dp[i][j] ==true )
      cnt <-  cnt+1


   return cnt



\end{lstlisting}

\bigskip\subsection{Algorithm 4 Pseudo Code}\label{AA}



\begin{lstlisting}[caption=Algorithm 4, style=chstyle, language=c++]

//=====================================
            //--PSEUDO CODE--//
//====================================


function main( )
 get string 
 long long int n <- string.size()
 long long int answer
 answer <- 0
 vector<long long int>d1(n)
 for i <- 0 to n,l <- 0,r <- -1 
      long long int  k <- (i > r) ? 1 : 
                min(d1[l + r - i], r - i + 1)
                
      while ( i - k >= 0 && i + k < n && 
                s[i - k] == s[i + k]) 
          k <- k+1
      end
      
      d1[i] <- k--
      if (i + k > r) 
         l <- i - k
         r <- i + k
 
 vector<long long int>d2(n)
 for i <- 0 to n,l <- 0,r <- -1 
      long long int  k <- (i > r) ? 0 : 
      min(d2[l + r - i+1], r - i + 1)
      
      while (i - k-1 >= 0 && i + k < n && 
            s[i - k-1] == s[i + k]) 
            k <- k+1
      end
      
      d1[i] <- k--
      if (i + k > r) 
        l <- i - k-1
        r <- i + k
           
    for i <- 0 to n
	answer←answer+d1[i]/2+1
	
    for i <- 0 to n
	answer <- answer+d2[i]/2
 
 print answer




\end{lstlisting}


\section*{ILLUSTRATION}

\begin{itemize}
\item Input : string = acca\\

Length of the given string = 4\\\\

        What tabulation table depicts :\\\\

        
        
\begin{center}
 

 \begin{tabularx}{0.4\textwidth} { 
  | >{\raggedright\arraybackslash}X 
  | >{\centering\arraybackslash}X 
  | >{\raggedleft\arraybackslash}X
  | >{\raggedleft\arraybackslash}X
  | }
  
  
   \hline
  a &   c  &  c  &   a \\   [0.5ex] 
 \hline\hline
   
   
 \hline
  
    a & ac & acc  & acca \\ 
 \hline
 
    0 & c &  cc &  cca \\
 \hline
 

    0
 &  0
   &  c
   &  ca

 \\
 \hline
 
    0
 &  0   
 &     0
 &    a

 \\ [1ex] 
 \hline
\end{tabularx}
\end{center}
              
  \bigskip 

\item First, we will create a 2d matrix dp[4][4] \\
Initially dp[][] matrix and cnt  be :\\\\

dp[4][4] = 
\begin{Bmatrix}
 0 & 0 &  0 &  0 \\
           	       0 &  0 &  0 &  0 \\
                      0 &  0 &  0 &  0 \\
                      0 &  0 &  0 &  0\\
\end{Bmatrix}

cnt = 0\\\\


\item Clearly, there are 4 diagonals which we will traverse.\\
Diagonal 1 : connecting (0,0) to (3,3)\\
Diagonal 2 : connecting (0,1) to (2,3)\\
Diagonal 3 : connecting (0,2) to (1,3)\\
Diagonal 4 : connecting (0,3) to (0,3)\\\\
 
 
 \item Now, using algorithm 2 for first diagonal,  the updated matrix and cnt will be :\\
Since “a” , “c”, “c”, “a” are palindromes, so mark true\\\\

dp[4][4] =\begin{Bmatrix}
 1 & 0 &  0 &  0 \\
           	       0 &  1 &  0 &  0 \\
                      0 &  0 &  1 &  0 \\
                      0 &  0 &  0 &  1\\
\end{Bmatrix}

cnt = 4\\





\item Now, for all other diagonals algorithm 2 implementation is demonstrated as :\\

Diagonal 2 :\\
For dp[0][1] :\\
Since , ‘a’ != ‘c’ so mark false\\
For dp[1][2] :\\
Since , ‘c’ == ‘c’ so mark true\\
For dp[2][3] :\\
Since , ‘c’ != ‘a’ so mark false\\
Updated dp[][] matrix and cnt :\\\\
dp[4][4] = \begin{Bmatrix}
 1 & 0 &  0 &  0 \\
           	       0 &  1 &  1 &  0 \\
                      0 &  1 &  1 &  0 \\
                      0 &  0 &  0 &  1\\
\end{Bmatrix}

So, cnt = 5\\\\

\item Diagonal 3 :\\
For dp[0][2] :\\
Since , ‘a’ != ‘c’ so mark false\\
For dp[1][3] :\\
Since , ‘c’ != ‘a’ so mark false\\\

Updated dp[][] matrix and cnt :\\
dp[4][4] = \begin{Bmatrix}
 1 & 0 &  0 &  0 \\
           	       0 &  1 &  1 &  0 \\
                      0 &  0 &  1 &  0 \\
                      0 &  0 &  0 &  1\\
\end{Bmatrix}
\\So, cnt = 5\\\\

\item Diagonal 4 :\\
For dp[0][3] :\\
Since, ‘a’ == ‘a’ so look for dp[0+1][3-1]\\
Since, dp[1][2] = true, so mark true\\\\

Updated dp[][] matrix and cnt :\\
dp[4][4] = \begin{Bmatrix}
 1 & 0 &  0 &  1 \\
           	       0 &  1 &  1 &  0 \\
                      0 &  0 &  1 &  0 \\
                      0 &  0 &  0 &  1\\
\end{Bmatrix}

So, cnt = 6\\\\


Output: 6.	\\\\
\bigskip		




\end{itemize}
\section{ALGORITHM ANALYSIS}
\bigskip\subsection{Time Complexity}
Here, \[ N = Length of given string\]
\section*{Recursion}

In this approach,for each recursive step we check if the substring is palindrome or not.If it is palindrome we add 1 and recurse for the remaining substrings for 
(i+1,j) and (i,j-1) and subtract the common palindrome.Similar Operations are done if the string is not palindrome except that we do not add 1.Since for each recursive call we make 3 recursive calls hence the total time complexity would be exponential i.e \( 3^N) \).
So,

\[T=O(3^N)\]
     

\bigskip

\section*{Two-loop Strategy}


In this approach,we consider two for-loops,
The first for-loop determines the start index of the substring and the second for-loop determines the end index of the substring.
Since it takes two for-loops to consider all the substrings the time complexity for computing the substrings would be O(N^2).

For every computed substring we also have to check if it’s a palindrome or not which requires a traversal of half the length of the substring,hence the time complexity would be O(N).
Since it is being done for all the substrings the total time complexity would be O(N^3);\bigskip
So,

 \[T=O(N*N*N)\]

\bigskip
\section*{Dynamic Programming}


In this approach, we are following some steps :\\
1) Traversing each row and column  of the upper half of the dp[][] matrix to check whether the substring that corresponds to that cell is palindrome or not.\\

Clearly, time complexity gets affected by the above step.
Assuming, time taken for comparisons to be  unit.\\
Let's suppose step 1 time complexity be T Since traversing each rows and columns ,\\
So, T = O(N * N) \\
\[T =  O(N*N)\]



\bigskip

Best Case Complexity: O(N*N) \\ 

Worst Case Complexity: O(N*N)\\

Average Case Complexity : O(N*N)

\bigskip
\section*{Manacher's Algorithm}

At the first glance it's not obvious that this algorithm has linear time complexity, because we often run the naive algorithm while searching the answer for a particular position.We can notice that every iteration of trivial algorithm increases r by one. Also r cannot be decreased during the algorithm. So, trivial algorithm will make O(n) iterations in total.
Also, other parts of Manacher's algorithm work obviously in linear time. Thus, we get O(n) time complexity.


\bigskip\subsection{Space Complexity}
Here, \[ N = Length of given string\]
\section*{Recursion algorithm}
Our memory complexity is determined by the number of return statements because each function call will be stored on the program stack. To generalize, a recursive function's memory complexity is O(recursion depth).Hence in this case space complexity would be O(N).

\bigskip\section*{Two-loop Strategy}

For the space analysis of the algorithm,some variables are used having constant space requirement.As far as the algorithm is concerned we do not use any extra array/string/container to find the result.Hence the overall space complexity is constant i.e O(k).

\bigskip\section*{Dynamic Programming}

Assuming, variables created don’t affect the space used calculation much.\\
Since we are creating a 2D matrix of the size  N X N, so we are using N*N space .Hence, the space complexity of the algorithm 2 will be O(N*N).

\bigskip\section*{Manacher's Algorithm}
For the space analysis of the algorithm we have used two vectors/arrays to store odd length and even even length palindromes apart from the input string and some variables having constant space requirement.Hence the overall space complexity is O(N).\bigskip


For the analysis of the algorithm, we break the code snippet into several segments and count the number of addition, subtraction, multiplication, division, assign, comparison and by considering all those operations as 1 unit of operation, we will try to find the overall time complexity of the algorithm by multiplying the total number of operations into the number of times the code segment is running in the algorithm. For the space complexity, we tend to count the total amount of space taken by that algorithm to implement and then find a relation between the amount of space required and the input. The time and space complexity analysis for various steps is as shown below 
\\

\bigskip
\section*{ Aposteriori analysis}\bigskip

\begin{tabularx}{0.4\textwidth} { | >{\raggedright\arraybackslash}X | >{\centering\arraybackslash}X | 
>{\raggedleft\arraybackslash}X | 
}
  
   
\hline
N ( String Length ) & Time (in ms) \\
   \hline
   

50 &   13.667\\
   \hline
   
100 & 12.928\\
   \hline
   
500 &   10.372\\
   \hline
   
1000 &   13.360\\
   \hline
   
5000 &   68.286\\
   \hline
   
8000 &  206.919\\
   \hline
   
10000 &   376.646\\
   \hline
   


\end{tabularx}
\bigskip





\begin{tikzpicture}
\begin{axis}[
    title={},
    xlabel={N},
    ylabel={Time},
     xmin=0, xmax=10000,
    ymin=0, ymax=400,
    xtick={1000,2000,3000,4000,5000,6000,7000,8000,9000},
    ytick={0,100,200,300,400},
    legend pos=north west,
    ymajorgrids=true,
    grid style=dashed,
]

\addplot[
    color=blue,
    mark=square,
    ]
    coordinates {
    (50,13.667)(100,12.928)(500,10.372)(1000,13.360)(5000,68.286)(8000,206.919)(10000,376.646)
    };
   
    
\end{axis}
\end{tikzpicture}

\section*{Figure1: Time complexity graph for aposteriori analysis. ( Keeping N on X-axis )
}

The graph represents Time v/s N ( string length )\\
\bigskip




\section{Conclusion}
The algorithms discussed in this paper can be used to find the number of palindromic substrings. In the matter of comparison between these two algorithms,in Approach 1, time complexity i.e \( O(3^N) \) and in approach 2 it would be \( O(N^3) \). Both would be higher as compared to Algorithm 3 i.e  \( (O(N^2)). \)\\
So we can conclude that Dynamic Programming Solution i.e the Algorithm 3 is optimal and efficient among them.
However, there exists a better standard algorithm named Manacher’s algorithm with time and space complexity of O(n). We have discussed about it also as algorithm 4.

\bigskip


\section{REFERENCES}

\bigskip

[1] \\
https://en.wikipedia.org/wiki/Dynamic\textunderscore programming\\

[2] \\
https://www.geeksforgeeks.org/tabulation-vs-memoization/ \\

[3] \\
https://www.ideserve.co.in/learn/time-and-space-complexity-of-recursive-algorithms\\

[4] \\
https://www.geeksforgeeks.org/auxiliary-space-recursive-functions/



\end{document}
